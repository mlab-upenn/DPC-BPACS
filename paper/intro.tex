\section{INTRODUCTION}

Efficient control of buildings requires high fidelity models that capture the evolution of the state of the building with time, for example, how the power consumption and zone temperature are affected when the chilled water or the supply air set points are changed with time or when outside weather conditions are different.
Model Predictive Control (MPC) uses such models to predict the state of the building over a finite horizon and optimize the performance with a given objective like load curtailment while meeting thermal comfort and operation constraints.

To this end, the classes of models that are most widely studied in the literature use first principles based on physics. 
These include the \textit{white box} models typically based on high fidelity simulation software like EnergyPlus (E+) \cite{Deru2011} and TRNSYS \cite{Transys1975}, and the \textit{grey box} models based on Resistor Capacitance (RC) networks \cite{Deng2010}.
The user expertise, time, and associated sensor costs required to develop such models of a single building are very high.
This is because such models require detailed information about the geometry of a building, design and equipment layout plans, material properties, and equipment and operational schedules. 
%There is always a gap between the modeled and the real building and the domain expert must then manually tune the model to match the measured data from the building.
Moreover, the modeling process also varies from building to building with the construction and types of installed equipment. 
%Another major downside with physics-based modeling is that enough data is not easily available and guesses for parameter values have to be made, which also requires expert know how.
After several years of work on using first principles based models for peak power reduction, and energy optimization for buildings, multiple authors \cite{Sturzenegger2016,vzavcekova2014} have concluded that the biggest hurdle to mass adoption of intelligent building control is the cost and effort required to capture accurate dynamical models of the buildings.

We take an alternative route to the physics-based approach, i.e.~\textit{black box} modeling based on machine learning algorithms to learn a digital twin of the underlying physical system, a building in this case.
Our approach reduces the cost and time to model the buildings by an order of magnitude \cite{JainICCPS2018,JainCDC2017,JainACC2017,nghiemetal16gp,behletal15dradvisor}.
We learn data-driven (D+) models using only historical data available via sensors already installed in the buildings -- thermostats, multimeters -- and historical weather data.
The D+ models can not only be used for prediction but also for real-time MPC.
These models are scalable and integrate seamlessly to the existing SCADA software or the building energy management systems (BEMS).

In this paper, we present an end-to-end architecture for modeling and HVAC control of large scale buildings efficiently using machine learning. 
We explain and demonstrate with examples our complete pipeline starting from tools for data extraction from existing SCADA/BEMS to learning accurate control-oriented data-driven models using Gaussian Processes (GP) to real-time predictive control with high confidence.