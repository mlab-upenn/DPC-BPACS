\section{MPC WITH GAUSSIAN PROCESSES}
\label{S:dpc}

This section presents a data-driven predictive control approach for buildings using GPs.
Suppose that GP models of a building's power demand and zone temperatures are already developed, as discussed in the previous section.
The predicted power demand at any future time $t+\tau$ in a window of $N$ time steps starting from the current time $t$, for \(\tau \in \{0,\dots,N-1\}\), is given by Equation~\eqref{eq:dpc:prediction}.
The output at time \(t+\tau\) depends upon %\(x_{t+\tau|t}\)  which is a function of
the control inputs \(u_{t+\tau-m}, \dots, u_{t+\tau}\).
We next show the flexibility of using these models in different applications.

\subsection{Demand Tracking Control}

\begin{figure*}[t]
  \centering
  \includegraphics[width=0.9\linewidth]{images/Dashboard-PowerTrack.png}
  \caption{Status of the LargeOffice as seen on SCADA system during Demand Tracking Control. The blue vertical line shows the current time of operation. Power consumption: the baseline power consumption is shown in dashed line, the curtailment is shown in solid green and the 95\% confidence bounds are shown in solid red. Since a fixed curtailment is tracked without thermal comfort constraints in problem \eqref{E:mpc:tracking}, the zone temperature shoots up by \(2^\circ C\).}
  \label{F:tracking}
\end{figure*}

This predictive control formulation fits demand response applications where a customer -- a building in this case -- must curtail its power demand by a certain amount from its baseline, or must track an Area Control Signal (ACS) sent from the grid operator.
For Demand Tracking Control we only require GP model \(\mathcal{M}_1\).
We are interested in solving the following optimization problem
\begin{align}
  &\minimize \sum_{\tau=0}^{N-1} (\bar{y}_{t+\tau} - y_{\mathrm{ref},t+\tau})^2 + \lambda \sigma^2_{y,t+\tau} \label{E:mpc:tracking} \\
  & 
    \begin{aligned}
      \text{subject to}\ \  & \bar{y}_{t+\tau} = \mu(x_{t+\tau}) + K_\star K^{-1} (Y - \mu(X)) \nonumber\\
      & \sigma^2_{y,t+\tau} = K_{\star \star} - K_\star K^{-1} K_\star^T \\
      & u_{t+\tau} \in \mathcal{U} \nonumber \\
      & \Pr(y_{t+\tau} \in \mathcal{Y}) \geq 1 - \epsilon \nonumber
    \end{aligned}
\end{align}
where the constraints hold for all \(\tau \in \{0,\dots,N-1\}\).
Here, \(K_\star = [k(x_{t+\tau}, x_1), \dots, k(x_{t+\tau}, x_N)]\), \(K_{\star \star} = k(x_{t+\tau}, x_{t+\tau})\). %, and $K$ is the covariance matrix with elements \(K_{ij} = k(x_i, x_j)\).
The signal $y_{\mathrm{ref}}$ is a reference power demand trajectory the building should follow as closely as possible.
The last constraint is probabilistic, which states that the building's power demand must stay inside a set $\mathcal{Y}$ with a probability of at least $1 - \epsilon$.
For instance, this constraint can control the quality of tracking the reference $y_{\mathrm{ref}}$.

We solve the optimization problem~\eqref{E:mpc:tracking} to compute optimal \(u_{t}^*, \dots, u_{t+N-1}^*\), apply \(u_{t}^*\) to the system and proceed to time \(t+1\).
Although all the constraints in \eqref{E:mpc:tracking} are of analytical forms, the optimization can be computationally hard to solve due to the high nonlinearity and computational burden of the GP.
% For the case study in Sec.~\ref{S:casestudy} we choose a combination of a squared exponential and rational quadratic kernel which results in nonconvex problem.
We use the nonlinear solver IPOPT \cite{Waechter2009b} and the optimization modeling framework CasADi \cite{Andersson2013b} to solve \eqref{E:mpc:tracking}.

We show the real-time results on the SCADA system in Figure \ref{F:tracking}.
The dashboard is configured to show the current power consumption, current setpoints, current weather conditions, a 12 hour history of the temperature of one of the zones (CoreMid), a 12 hour history of the setpoints -- cooling temperature (red), supply air temperature (green), chiller water temperature (blue) -- and a 12 hour history of the power consumption (solid green) along with a 12 hour forecast of the baseline consumption (dashed).
In this example, the controller follows a rule-based strategy until 3 pm.
A Demand Response is scheduled between 3-5 pm when D+ controller based on \eqref{E:mpc:tracking} is used.
The controller provides a sustained curtailment of 100kW from the baseline.
Since a fixed curtailment is desired without thermal comfort constraints, the zone temperature shoots up by \(2^\circ C\).

\subsection{Climate Control with Minimum Energy Usage}

\begin{figure*}[t]
  \centering
  \includegraphics[width=0.9\linewidth]{images/Dashboard-DR.png}
  \caption{Status of the LargeOffice as seen on SCADA system during Climate Control with Minimum Energy Usage. The blue vertical line shows the current time of operation. Power consumption: the baseline power consumption is shown in dashed line, the curtailment is shown in solid green and the 95\% confidence bounds are shown in solid red. Zone temperature: the thermal comfort constraint keeps the temperature between a desired range \(23-25^\circ C\) defined in the optimization problem \eqref{E:mpc:climate}.}
  \label{F:climate}
\end{figure*}

Other interesting objective for energy management is of minimizing energy usage while meeting thermal comfort and operation constaints. 
This is always desired since the consumers want to minimize their electricity bills.
When the real-time price of electricity peaks, the utilities may also ask their customers to minimize their consumption between a specified time based on the available flexibility.

In this example, we require both GP models \(\mathcal{M}_1\) and \(\mathcal{M}_2\).
Specifically, the model \(\mathcal{M}_2\) is used to enforce thermal comfort constraints in the optimization problem below:
\begin{align}
&\minimize \ \sum_{\tau=0}^{N-1} \bar{y}_{t+\tau} + \lambda \sigma^2_{y,t+\tau} \label{E:mpc:climate} \\
& 
\begin{aligned}
\text{subject to}\ \  & \begin{rcases}\bar{y}_{t+\tau} = \mu(x_{t+\tau}) + K_\star K^{-1} (Y - \mu(X)) \nonumber\\
\sigma^2_{y,t+\tau} = K_{\star \star} - K_\star K^{-1} K_\star^T
\end{rcases} \text{GP } \mathcal{M}_1 \\
& \begin{rcases}
 \bar{T}_{t+\tau} = \mu(x_{t+\tau}) + K_\star K^{-1} (Y - \mu(X)) \nonumber\\
\sigma^2_{T,t+\tau} = K_{\star \star} - K_\star K^{-1} K_\star^T \\
\Pr(T_{\mathrm{min}} \leq T_{t+\tau} \leq T_{\mathrm{max}}) \geq 1 - \epsilon \nonumber
\end{rcases} \text{GP } \mathcal{M}_2 \\
& \ u_{t+\tau} \in \mathcal{U} \nonumber
\end{aligned}
\end{align}
Here, \(\bar{y},\sigma^2_{y}\) denote the mean and variance of the GP \(\mathcal{M}_1\), and \(\bar{T},\sigma^2_{T}\) of the GP \(\mathcal{M}_2\). 
The goal is to optimize the inputs \(u\) that minimize the power consumption while maintaining thermal comfort with high probability. 
We again solve the optimization problem~\eqref{E:mpc:climate} to compute optimal \(u_{t}^*, \dots, u_{t+N-1}^*\), apply \(u_{t}^*\) to the system and proceed to time \(t+1\).

We consider a Demand Response scenario when the utility company notifies this building to minimize the power consumption between 11am - 2pm.
Between 2pm - 3pm, the controller switches to tracking the baseline consumption based on \eqref{E:mpc:tracking} to prevent a sudden kickback.
The status of the SCADA system at 3pm is shown in Figure \ref{F:climate}.
The controller exploits the available energy flexibility in providing maximum curtailment while ensuring that the zone temperature is always between the prescribed range \(23-25^\circ C\).



%%% Local Variables:
%%% mode: latex
%%% TeX-master: "main"
%%% End:
