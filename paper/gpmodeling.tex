\section{MODELING WITH GAUSSIAN PROCESSES}
\label{S:gp}

In this section, we briefly introduce modeling with Gaussian Process (GP) and its applications in control.
More details can be found in \cite{Rasmussen2006} %on GP for machine learning and in
and \cite{Kocijan2016}. % on GP modeling of dynamic systems.

\begin{definition}[\cite{Rasmussen2006}]
A Gaussian Process is a collection of random variables, any finite number of which have a joint Gaussian distribution.
\end{definition}
Consider noisy observations \(y\) of an underlying function \(f: \RR^n \mapsto \RR\) through a Gaussian noise model: \(y = f(x) + \GaussianDist{0}{\sigma_n^2}\), \(x \in \RR^n\).
A GP of \(y\) is fully specified by its mean function \(\mu(x)\) and covariance function \(k(x,x')\),
\begin{align}
\label{E:gp:prior}
\mu(x; \theta) &= \EE [f(x)] \\
k(x,x'; \theta) &= \EE [(f(x)\!-\!\mu(x)) (f(x') \!-\! \mu(x'))] + \sigma_n^2 \delta(x,x') \nonumber
\end{align}
where \(\delta(x,x')\) is the Kronecker delta function.
The hyperparameter vector \(\theta\) parameterizes the mean and covariance functions.
This GP is denoted by \(y \sim \mathcal{GP}(\mu, k; \theta)\).

Given the regression vectors \(X = [x_1, \dots, x_N]^T\) and the corresponding observed outputs \(Y = [y_1, \dots, y_N]^T\), we define training data by $\D = (X, Y)$. The distribution of the output \(y_\star\) corresponding to a new input vector \(x_\star\) is a Gaussian distribution \(\GaussianDist{\bar{y}_\star}{\sigma_\star^2}\), with mean and variance given by
\begin{subequations}
\label{E:gp-regression}
\begin{align}
\bar{y}_\star &= g_{\mathrm{m}} (x_{\star}) \coloneqq \mu(x_\star) + K_\star K^{-1} (Y - \mu(X))\\
\sigma_\star^2 &= g_{\mathrm{v}} (x_{\star}) \coloneqq K_{\star \star} - K_\star K^{-1} K_\star^T \text,
\end{align}
\end{subequations}
where \(K_\star = [k(x_\star, x_1), \dots, k(x_\star, x_N)]\), \(K_{\star \star} = k(x_\star, x_\star)\), and $K$ is the covariance matrix with elements \(K_{ij} = k(x_i, x_j)\).

Note that the mean and covariance functions are parameterized by the hyperparameters $\theta$, which can be learned by maximizing the likelihood: \(\argmax_\theta \Pr(Y \vert X, \theta)\).
The covariance function \(k(x,x')\) indicates how correlated the outputs are at \(x\) and \(x'\), with the intuition that the output at an input is influenced more by the outputs of nearby inputs in the training data $\D$.
In other words, a GP model specifies the structure of the covariance matrix of, or the relationship between, the input variables rather than a fixed structural input--output relationship.
It is therefore highly flexible and can capture complex behavior with fewer parameters.
An example of GP prior and posterior is shown in Fig.~\ref{F:gp:prior:posterior}. We use a constant mean function and a combination of squared exponential kernel and rational quadratic kernel as described in Sec.~\ref{SS:casestudy:gp}.
There exists a wide range of covariance functions and combinations to choose from \cite{Rasmussen2006}. 

GPs offer several advantages over other machine learning algorithms that make them more suitable for identification of dynamical systems.
\begin{enumerate}
\item GPs provide an estimate of uncertainty or confidence in the
  predictions through the predictive variance.  While the predictive mean is often used as the best guess of the output, the full distribution can be used in a meaningful way. For example, we can estimate a 95\% confidence bound for the predictions which can be used to measure control performance.
\item GPs work well with small data sets.  This ability is generally useful for any learning application.
\item GPs allow including prior knowledge of the system behavior by defining priors on the hyperparameters or constructing a particular structure of the covariance function.  This feature enables incorporating domain knowledge into the GP model to improve its accuracy.
\end{enumerate}

\subsection{Gaussian Processes for Dynamical Systems}
\label{SS:intro-gp:control}

\begin{figure}[!tb]
	\centering
	\setlength\fwidth{0.36\textwidth}
	\setlength\hwidth{0.18\textwidth}	
	% This file was created by matlab2tikz.
%
%The latest updates can be retrieved from
%  http://www.mathworks.com/matlabcentral/fileexchange/22022-matlab2tikz-matlab2tikz
%where you can also make suggestions and rate matlab2tikz.
%
\definecolor{mycolor1}{rgb}{0.97647,0.89804,1.00000}%
%
\begin{tikzpicture}

\begin{axis}[%
width=\fwidth,
height=0.982\hwidth,
at={(0\fwidth,0\hwidth)},
scale only axis,
xmin=1,
xmax=48,
xtick={8,16,24,32,40},
xticklabels={{2am},{4am},{6am},{8am},{10am}},
ymin=13.8495974547768,
ymax=404.39922079824,
ylabel style={font=\color{white!15!black}},
ylabel={power [kW]},
axis background/.style={fill=white},
legend style={at={(0.5,1.03)}, anchor=south, legend columns=4, legend cell align=left, align=left, fill=none, draw=none},
xlabel style={font=\footnotesize},ylabel style={font=\footnotesize},legend style={font=\footnotesize},ticklabel style={font=\footnotesize},ylabel shift = -5 pt,xlabel shift = -5 pt,
]

\addplot[area legend, draw=white!75!gray, fill=white!75!gray]
table[row sep=crcr] {%
x	y\\
1	386.646965191719\\
2	386.646965191719\\
3	386.646965191719\\
4	386.646965191719\\
5	386.646965191719\\
6	386.646965191719\\
7	386.646965191719\\
8	386.646965191719\\
9	386.646965191719\\
10	386.646965191719\\
11	386.646965191719\\
12	386.646965191719\\
13	386.646965191719\\
14	386.646965191719\\
15	386.646965191719\\
16	386.646965191719\\
17	386.646965191719\\
18	386.646965191719\\
19	386.646965191719\\
20	386.646965191719\\
21	386.646965191719\\
22	386.646965191719\\
23	386.646965191719\\
24	386.646965191719\\
25	386.646965191719\\
26	386.646965191719\\
27	386.646965191719\\
28	386.646965191719\\
29	386.646965191719\\
30	386.646965191719\\
31	386.646965191719\\
32	386.646965191719\\
33	386.646965191719\\
34	386.646965191719\\
35	386.646965191719\\
36	386.646965191719\\
37	386.646965191719\\
38	386.646965191719\\
39	386.646965191719\\
40	386.646965191719\\
41	386.646965191719\\
42	386.646965191719\\
43	386.646965191719\\
44	386.646965191719\\
45	386.646965191719\\
46	386.646965191719\\
47	386.646965191719\\
48	386.646965191719\\
48	31.6018530612979\\
47	31.6018530612979\\
46	31.6018530612979\\
45	31.6018530612979\\
44	31.6018530612979\\
43	31.6018530612979\\
42	31.6018530612979\\
41	31.6018530612979\\
40	31.6018530612979\\
39	31.6018530612979\\
38	31.6018530612979\\
37	31.6018530612979\\
36	31.6018530612979\\
35	31.6018530612979\\
34	31.6018530612979\\
33	31.6018530612979\\
32	31.6018530612979\\
31	31.6018530612979\\
30	31.6018530612979\\
29	31.6018530612979\\
28	31.6018530612979\\
27	31.6018530612979\\
26	31.6018530612979\\
25	31.6018530612979\\
24	31.6018530612979\\
23	31.6018530612979\\
22	31.6018530612979\\
21	31.6018530612979\\
20	31.6018530612979\\
19	31.6018530612979\\
18	31.6018530612979\\
17	31.6018530612979\\
16	31.6018530612979\\
15	31.6018530612979\\
14	31.6018530612979\\
13	31.6018530612979\\
12	31.6018530612979\\
11	31.6018530612979\\
10	31.6018530612979\\
9	31.6018530612979\\
8	31.6018530612979\\
7	31.6018530612979\\
6	31.6018530612979\\
5	31.6018530612979\\
4	31.6018530612979\\
3	31.6018530612979\\
2	31.6018530612979\\
1	31.6018530612979\\
}--cycle;
\addlegendentry{$\text{prior }\mu\text{ }\pm\text{ 2}\sigma$}

\addplot [color=black, dashed, line width=1.0pt]
  table[row sep=crcr]{%
1	209.124409126508\\
48	209.124409126508\\
};
\addlegendentry{$\text{prior }\mu$}


\addplot[area legend, draw=mycolor1, fill=mycolor1]
table[row sep=crcr] {%
x	y\\
1	154.709201629968\\
2	166.075918612389\\
3	139.2395727394\\
4	150.627198219011\\
5	156.562639738461\\
6	172.537252098066\\
7	149.736958849622\\
8	122.449211106838\\
9	137.147704214366\\
10	127.377735086393\\
11	143.923100614629\\
12	154.245932548526\\
13	122.463825041785\\
14	172.450813790158\\
15	162.394161326987\\
16	153.102756456198\\
17	130.944901359138\\
18	218.014957344162\\
19	247.60829556031\\
20	218.774750045423\\
21	232.998363083984\\
22	270.795112821555\\
23	268.724584385649\\
24	290.95948553084\\
25	300.682967077219\\
26	294.875735172657\\
27	314.757769779285\\
28	313.247180951257\\
29	347.323672820772\\
30	296.05671917331\\
31	301.042057704437\\
32	267.702473452336\\
33	273.088289651178\\
34	209.312427630456\\
35	216.802255898329\\
36	268.633216087559\\
37	228.988631430966\\
38	213.423993131178\\
39	183.181633964876\\
40	201.306696759562\\
41	204.566034839882\\
42	238.727065874924\\
43	255.945408503762\\
44	253.536624332962\\
45	272.918841238678\\
46	229.369289663929\\
47	227.109791028515\\
48	192.638566318614\\
48	156.024126508636\\
47	190.753621171366\\
46	193.032780488085\\
45	236.383047496946\\
44	215.860169346718\\
43	219.236377026459\\
42	202.616050681686\\
41	168.172667959549\\
40	165.052675911654\\
39	145.557971092335\\
38	176.467689053364\\
37	191.914942913949\\
36	231.696582339222\\
35	179.351228779518\\
34	173.400552712333\\
33	237.031606471281\\
32	230.183454657379\\
31	263.557792224223\\
30	258.261638782427\\
29	308.397338208832\\
28	274.619563878076\\
27	276.795039204419\\
26	255.661011500426\\
25	261.710731210065\\
24	252.646146479488\\
23	231.178525931647\\
22	234.338683404564\\
21	195.569106888561\\
20	180.844093749074\\
19	210.380218922248\\
18	182.080400287927\\
17	95.6282421712857\\
16	117.359884257066\\
15	126.407916512676\\
14	136.673795119128\\
13	86.5352281989955\\
12	117.899804518827\\
11	106.897592885672\\
10	89.6440805991176\\
9	99.1791317367729\\
8	85.6454991550155\\
7	112.808368940314\\
6	136.547182527759\\
5	120.805709082496\\
4	113.744125160121\\
3	102.833336478683\\
2	129.510389380878\\
1	118.072007567044\\
}--cycle;
\addlegendentry{$\text{posterior }\mu\text{ }\pm\text{ 2}\sigma$}

\addplot [color=red]
  table[row sep=crcr]{%
1	136.390604598506\\
2	147.793153996633\\
3	121.036454609042\\
4	132.185661689566\\
5	138.684174410479\\
6	154.542217312913\\
7	131.272663894968\\
8	104.047355130927\\
9	118.163417975569\\
10	108.510907842755\\
11	125.410346750151\\
12	136.072868533677\\
13	104.49952662039\\
14	154.562304454643\\
15	144.401038919831\\
16	135.231320356632\\
17	113.286571765212\\
18	200.047678816045\\
19	228.994257241279\\
20	199.809421897248\\
21	214.283734986272\\
22	252.566898113059\\
23	249.951555158648\\
24	271.802816005164\\
25	281.196849143642\\
26	275.268373336542\\
27	295.776404491852\\
28	293.933372414666\\
29	327.860505514802\\
30	277.159178977869\\
31	282.29992496433\\
32	248.942964054857\\
33	255.059948061229\\
34	191.356490171394\\
35	198.076742338924\\
36	250.164899213391\\
37	210.451787172458\\
38	194.945841092271\\
39	164.369802528606\\
40	183.179686335608\\
41	186.369351399715\\
42	220.671558278305\\
43	237.590892765111\\
44	234.69839683984\\
45	254.650944367812\\
46	211.201035076007\\
47	208.931706099941\\
48	174.331346413625\\
};
\addlegendentry{$\text{posterior }\mu$}

\end{axis}
\end{tikzpicture}%
	\caption{Example of priors calculated using \eqref{E:gp:prior} and posteriors using \eqref{E:gp-regression} for predicting power consumption of a building for \(12\) hrs. Initially the mean is constant because \(\mu(x)\) is constant, and we observe a high variance. The posterior agrees with the actual power consumption with high confidence.}
	\label{F:gp:prior:posterior}
\end{figure}

GPs can be used for modeling nonlinear dynamical systems, by feeding autoregressive, or time-delayed, input and output signals back to the model as regressors \cite{Kocijan2016}.
Specifically, in control systems, it is common to use an autoregressive GP to model a dynamical system represented by the nonlinear function
\begin{math}
y_{t} = f(x_t)
\end{math}
where
\begin{equation*}
x_{t}\!=\![y_{t-l}, \dots, y_{t-1}, u_{t-m}, \dots, u_t, w_{t-p}, \dots, w_{t-1}, w_t] \text.
\end{equation*}
Here, \(t\) denotes the time step, \(u\) the control input, \(w\) the exogenous disturbance input, \(y\) the (past) output, and \(l\), \(m\), and \(p\) are respectively the lags for autoregressive outputs, control inputs, and disturbances.
Note that \(u_t\) and \(w_t\) are the current control and disturbance inputs.
The vector of all autoregressive inputs can be thought of as the current state of the model.
A dynamical GP can then be trained from data in the same way as any other GPs.

When a GP is used for control or optimization, it is usually necessary to simulate the model over a finite number of future steps and predict its multistep-ahead behavior.
Because the output of a GP is a distribution rather than a point estimate, the autoregressive outputs fed to the model beyond the first step are random variables, resulting in more and more complex output distributions as we go further.
Therefore, a multistep simulation of a GP involves the propagation of uncertainty through the model.
There exist several methods for uncertainty propagation in GPs \cite{Kocijan2016}.

It was shown in \cite{nghiemetal16gp} that the \emph{zero-variance method}, which replaces the autoregressive outputs with their corresponding expected values and therefore does not propagate uncertainty, could achieve sufficient prediction accuracy compared to the Monte-Carlo method of uncertainty propagation.
Its computational simplicity is attractive, especially in optimization applications where the GP must be simulated for many time steps.
Consequently, the zero-variance method was selected for predicting future outputs in this work.


%\begin{enumerate}
%	\item The \emph{Monte-Carlo method} obtains samples of the output distribution under input uncertainty, which can be seen as a Gaussian mixture.  This Gaussian mixture becomes more complex in later steps of the simulation, therefore efficient numerical algorithms must be implemented.  This method can achieve good prediction accuracy at the expense of high computational load.  It is also general, \ie it can be used with any covariance functions.
%	\item The \emph{zero-variance method} does not propagate uncertainty.  At each step, the autoregressive outputs are replaced by their corresponding expected values.  Obviously, this method will underestimate the variances of the output distributions.  However, its computational simplicity is attractive, especially in optimization applications where the GP must be simulated for many times.  In such cases, if the prediction error caused by not propagating uncertainty is insignificant, the zero-variance method can and should be used.  For more detailed discussions on this topic, see \cite{Kocijan2016,girard04approximate}.
%\end{enumerate}



%%% Local Variables:
%%% mode: latex
%%% TeX-master: "main"
%%% End:
