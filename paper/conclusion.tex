\section{CONCLUSION}

This paper presented a set of methods and tools to incorporate ``digital twins'' of real buildings into existing SCADA systems.
We proposed a data-driven modeling approach using Gaussian Process to quickly and inexpensively capture a model of a building solely from its measurement data.
This type of models, called D+ models, together with EnergyPlus (E+) models of buildings serve as ``digital twins'' of the buildings.
In addition to requiring significantly lower cost and effort to develop, compared to E+ models, D+ models are more suitable for Model Predictive Control (MPC) and more adaptive to changes in the buildings.
Two applications of MPC with D+ models were formulated for demand-tracking control and climate control with minimum energy.

We developed an EnergyPlus-Python bridge, called pyEp, to interface Python code with EnergyPlus to perform co-simulations, which is useful for implementing advanced algorithms such as machine-learning-based modeling and optimization-based control.
We also developed an EnergyPlus-OPC bridge, which completes our toolchain for integrating E+ and D+ models into SCADA systems.
Through OPC tag mappings, these digital twins can directly exchange data with a SCADA system, receiving control commands and returning measurement values as if they were real buildings.
For the first time, our toolchain has enabled seamless real-time in-the-loop prediction and advanced control of both software buildings and physical buildings within the same SCADA environment.
The toolchain was demonstrated in a case study, which showed the effectiveness of both our software and our proposed data-driven MPC approach for buildings.

\todo[inline]{Well, any idea whether and how we want to extend this work?}

%%% Local Variables:
%%% mode: latex
%%% TeX-master: "main"
%%% End:
